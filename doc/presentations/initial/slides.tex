\documentclass[xcolor=dvipsnames]{beamer}
\usetheme{Warsaw}
\usecolortheme[named=RoyalBlue]{structure}
\usenavigationsymbolstemplate{}

\usepackage{ngerman}
\usepackage[utf8]{inputenc}
\usepackage{amssymb}
\usepackage{fancyvrb}
\usepackage{graphicx}
\usepackage{floatflt}
\usepackage{setspace}
\usepackage[absolute,overlay]{textpos}

\setbeamercolor{framesource}{fg=gray}
\setbeamerfont{framesource}{size=\tiny}

\newcommand{\myarrowl}{$\leftarrow$\hspace{0.01\textwidth}}
\newcommand{\myarrowr}{$\rightarrow$\hspace{0.01\textwidth}}

\begin{document}
\title{Akari goes tablet}
\author{Ivan Hell, Julian Lang}
\author[Ivan Hell, Julian Lang]{Ivan Hell, Julian Lang{\\\vspace*{10mm}\small Supervisor: Cynthia Kop}}

\begin{frame}
\titlepage
\end{frame}

\begin{frame}
  \frametitle{Content}
  \scriptsize 
  \tableofcontents
\end{frame}


%%-------------------- Introduction --------------------%%
\section{Introduction}
\begin{frame}
  \frametitle{What is Akari?}
  \begin{itemize}
  \item Japanese pencil puzzle game
  \item Also called ``Light Up''
  \item Published in 2001 by Nikoli, a publisher for games and logic puzzles
  \end{itemize}
\end{frame}

%%-------------------- Game-Rules  --------------------%%
\section{Game-rules}
\begin{frame}
  \frametitle{Game-field}
  \begin{itemize} 
  \item Rectangular grid of white and black cells
  \item Goal: Light bulbs must be placed to light all white cells
  \end{itemize}
\only<1>
{
  \begin{figure}[b]
     \includegraphics[width=0.5\textwidth]{img/empty.png}
  \end{figure}
}

\only<2>
{
  \begin{figure}[b]
     \includegraphics[width=0.5\textwidth]{img/solved.png}
  \end{figure}
}



\end{frame}

\begin{frame}
  \frametitle{Light bulbs}
  \begin{itemize}
  \item Can be placed within the white cells
  \item A bulb sends rays of light horizontally and vertically, illuminating its entire row and column unless its light is blocked by a black cell
  \item Bulbs may not shine on each other
  \end{itemize}
\only<1>
{
  \begin{figure}[b]
     \includegraphics[width=0.4\textwidth]{img/notfilled.png}
  \end{figure}
}

\only<2>
{
  \begin{figure}[b]
     \includegraphics[width=0.4\textwidth]{img/collision.png}
  \end{figure}
}
\end{frame}

\begin{frame}
  \frametitle{Black cells}
  \begin{itemize}
  \item May have a number from 0 to 4 on it 
  \item The number indicates the exact amount of bulbs that must be placed next to the cell (horizontally and vertically)
  \item Unnumbered black cells may have an arbitrary number of adjacent light bulbs
  \end{itemize}

\only<1>
{
  \begin{figure}[b]
     \includegraphics[width=0.4\textwidth]{img/block1.png}
  \end{figure}
}

\only<2>
{
  \begin{figure}[b]
     \includegraphics[width=0.4\textwidth]{img/block2.png}
  \end{figure}
}

\only<3>
{
  \begin{figure}[b]
     \includegraphics[width=0.4\textwidth]{img/block3.png}
  \end{figure}
}

\only<4>
{
  \begin{figure}[b]
     \includegraphics[width=0.4\textwidth]{img/block4.png}
  \end{figure}
}
\end{frame}



%%-------------------- Aim of the project --------------------%%
\section{Aim of the project}
\begin{frame}
  \frametitle{Aim of the project}
  \begin{itemize}
  \item Implement a user-interface for Android tablets
    \begin{itemize}
    \item User-friendly
    \item Support as many Android versions as possible
    \end{itemize}
  \item Implement a solver and generator for Akari puzzles
    \begin{itemize}
    \item Use encodings in satisfiability formulas
    \item Generate uniquely solvable puzzles of various difficulty 
    \end{itemize}
  \end{itemize}
\end{frame}

%%-------------------- Ideas --------------------%%
\section{Ideas for the Project}
\begin{frame}
  \frametitle{Possible engines}
  \begin{itemize}
  \item Cocos2D-x
    \begin{itemize}
    \item Platforms: iOS, Android, WindowsPhone8, BlackBerry
    \item Language: C++
    \item License: MIT
    \end{itemize}
  \item AndEngine
    \begin{itemize}
    \item Platforms: Android
    \item Language: Java
    \item License: LGPL
    \end{itemize}
  \item LibGDX
    \begin{itemize}
    \item Platforms: Android, iOS
    \item Language: Java
    \item License: Apache v2
    \end{itemize}
  \end{itemize}
\end{frame}

\begin{frame}
  \frametitle{Solver}
  \begin{itemize}
  \item Akari was proved to be NP-hard
  \item SAT solver could provide a more efficient solution then traditional approaches
  \item Possible solver:
   \begin{itemize}
     \item Sat4j
     \item Native compiled SAT solver
     \item SMT
   \end{itemize}
  \end{itemize}
\end{frame}


\begin{frame}
  \frametitle{Generator}
  \begin{itemize}
  \item For many puzzles the task to find an other solution for a already computed solution is also NP-hard 
  \item Creating random levels with a SAT solver could be tricky
  \item Finding unique solutions could be difficult
   \begin{itemize}
     \item Might be hard to express in SAT
     \item Converting a puzzle with multiple solutions into a puzzle with one solution could be impossible
     \item Solving of puzzles is a sub task of the generator \myarrowr  Testing many levels for unique satisfiability could be inefficient
   \end{itemize}
  \end{itemize}
\end{frame}


\begin{frame}
  \frametitle{Optional features}
  \begin{itemize}
  \item Level-Chooser that is able to synchronize with a server
  \item Give hints to the player
  \item Compare high-score with other players
  \item Provide multiple game-modes (standard-mode, time-mode, expert-mode, ...)
  \item Share random generated levels
  \end{itemize}
\end{frame}




\end{document}
